
% Para poner una foto-------------------
\begin{figure}[H]
    \centering
    \includegraphics[tamaño]{Fotos/logo fiuba.png}
    \caption{xxxxxxx}
\end{figure}

% Armar Tabla---------------------------
\begin{table}[H]
    \centering
    \begin{tabular}{| c | c | c |}\hline
    \textbf{Punto Crítico} & \textbf{Estabilidad} & \textbf{Autovalores} \\ \hline
    Nodo & Inestable & $\lambda_{1} >\lambda_{2} > 0$           \\ \hline
    Espiral & Estable &$\lambda = a +bi, a=0$           \\ \hline
\end{tabular}
    \\Tabla 1: Autovalores segun punto crítico y estabilidad.
\end{table}

%PARA QUE APAREZCAN LAS SUBSECCIONES CON LETRAS Y NO CON NUMEROS 
\renewcommand{\thesubsection}{\thesection.\alph{subsection}}


% Formato textos---------------
\textbf{}
\textit{}
\textsc{}
\underline
\acute{a}
\grave{a}
\tilde{a}
\bar{a} 
\hat{a} %^
\vec{a} %flecha
\dot{a} %puntito arriba de a
\ddot{a}
\dddot{a} 
\ul{} %subrayar
\texttt{} %maquina d escribir

%Simbolos:
https://manualdelatex.com/simbolos


% Ecuaciones-------------------
\begin{equation}
    % Fracciones
        \frac{120}{527}
        \left(\frac{5}{15}\right) %para que se acomode el parentesis
    % Raices
        \sqrt[4]{16}
    % Subindice
        y_{a}
    % Superindice
        y^a
    %llaves
        \{8\}
        |3|
        \|5\| % ||5||
        \langle 7 \rangle %<7>
    %Sumatoria
        \sum_{n=1}^{10}n
    %Productoria
        \prod_{n=1}^{10}n
    %Integrales
        \int_0^{\infty} x \mathrm{d}x
        \iint %(doble) \iiint (triple) etc...
    %Limites 
        \lim_{x\rightarrow\infty}\frac{3+x}{x^2}
    %Matrices
        \begin{matrix}
        5 & 4 & 8 \\
        4 & 0 & 7 \\
        3 & 5 & 6
        \end{matrix}
        %pmatrix=parentesis; bmatrix=corchetes[]; bmatrix=llaves{};vmatrix= modulo ||; Vmatrix= norma ||x|| 
    %Vectores
        \vec{x} %me pone una flechita en la v
        \hat{x} %me pone el sombrerito en la v
        \vec{u}\cdot \vec{v} %producto escalar
        \vec{u}\times \vec{v} %producto vectorial
    %Texto en la ecuacion
        F = ma \quad\text{Segunda ley de Newton}
        f(x) = 2x \quad\mathrm{si}\quad x < 2
    %Laplace; Fourier; etc..
        \mathcal{F}
        \mathcal{L}
    %Conjunto de numeros
        \mathbb{N}
        \mathbb{Z}
    %Letra gotica
         \mathfrak 
         
    \label{eq:nombre ecuacion}
\end{equation}

%Para referenciar la ecuacion pongo:
ecuación~\ref{eq:label equation}.


% Notas al pie de pagina

bla bla bla\footnote{Detalles interesantes de esa parte}
